\documentclass[16pt]{article}
\usepackage[utf8]{inputenc}
\usepackage[russian]{babel}
\usepackage{a4wide}
\usepackage{graphicx}
\usepackage{amsmath}
\usepackage{amssymb}
\usepackage{mathrsfs}  
\usepackage{amsthm}
\usepackage{import}
\usepackage{xifthen}
\usepackage{pdfpages}
\usepackage{transparent}
\usepackage{multicol}
\usepackage{indentfirst}
\usepackage{bbm}

\newcommand{\incfig}[2]{%
    \def\svgwidth{#2 mm}
    \import{./figures/}{#1.pdf_tex}
}

\newtheorem{Th}{Теорема}
\newtheorem{Lem}{Лемма}
\newtheorem{Def}{Определение}
\newtheorem*{Cons}{Следствие}
\newtheorem{Rem}{Замечание}
\newtheorem{St}{Утверждение}
\newenvironment{Proof}{\par\noindent{\bf Доказательство. \\}}{\hfill$\scriptstyle\blacksquare$}

\DeclareMathOperator{\Arccos}{arccos}
\DeclareMathOperator{\Arctg}{arctg}
\DeclareMathOperator{\Cl}{Cl}
\DeclareMathOperator{\Sign}{sgn}
\DeclareMathOperator*{\Argmax}{Argmax}
\DeclareMathOperator*{\Var}{Var}
\DeclareMathOperator*{\Min}{min}


\newcommand\Real{\mathbb{R}} 
\newcommand\A{(\cdot)} 
\newcommand\Sum[2]{\sum\limits_{#1}^{#2}}
\newcommand\Scal[2]{\langle #1,\, #2 \rangle}
\newcommand\Norm[1]{\left\| #1 \right\|}
\newcommand\Int[2]{\int\limits_{#1}^{#2}}
\newcommand{\Me}{\mathbb{E}}
\newcommand{\I}{\mathbbm{1}}
\newcommand{\Prb}{\mathbb{P}}
\newcommand\Reff[1]{(\ref{#1})}
\newcommand{\deq}{\overset{d}{=}}
\newcommand{\Bor}{\mathcal{B}}

\begin{document}
\thispagestyle{empty}

\begin{center}
\ \vspace{-3cm}

\includegraphics[width=0.5\textwidth]{msu.eps}\\
{\scshape Московский государственный университет имени М.~В.~Ломоносова}\\
Факультет вычислительной математики и кибернетики\\
Кафедра системного анализа

\vfill

{\LARGE Отчёт по практикуму}

\vspace{1cm}

{\Huge\bfseries <<Эллипсоидальные оценки для множества разрешимости>>}
\end{center}

\vspace{1cm}

\begin{flushright}
  \large
  \textit{Студент 415 группы}\\
  К.\,И.~Салихова

  \vspace{5mm}

  \textit{Руководитель практикума}\\
  к.ф.-м.н., доцент И.\,В.~Востриков
\end{flushright}

\vfill

\begin{center}
Москва, 2020
\end{center}

\newpage
\tableofcontents
\newpage
\section{Постановка задачи}
Рассматривается линейная управляемая система:

\begin{equation}\label{task}
\begin{cases}
\dot{x}(t)  = A(t)x(t) + B(t)u(t), \ \ t \in [t_0, t_1],\\
x(t_1) \in \mathcal{E}(x_1, X_1),\\
u(t)\in \mathcal{E}(p(t), P(t)).
\end{cases}
\end{equation}


Здесь $u(t) \in \Real^m, \, x(t) \in \Real^n,\, x_1 \in \Real^n, \,X_1 \in \Real^{n \times n},\, X_1 = X_1^* > 0, \, A(t) \in \Real^{n \times n}, \,B(t) \in \Real^{n \times m},\, p(t) \in \Real^m, \, P(t) \in \Real^{m \times m},\, P(t) = P^*(t) > 0 $ при любых $t$ из $[t_0,\, t_1]$. Функции $A(t),\, B(t),\,$ $p(t),\, P(t)$ непрерывны на $[t_0, t_1].$


Необходимо построить:
\begin{enumerate}
	\item Внутренние и внешние эллипсоидальные оценки множества разрешимости системы (\ref{task}).
	\item Проекции множества разрешимости на двумерную плоскость.
	\item Трехмерную проекцию множества разрешимости.
	
\end{enumerate}

\section{Внутренние и внешние оценки для суммы эллипсоидов}

Обозначим эллипсоид с центром $q \in \Real^n$ и матрицей конфигурации $Q \in \Real^{n \times n},\ Q = Q^* > 0$ 

$$\mathcal{E}(q, Q) = \{x\colon \Scal{(x - q)}{Q^{-1}(x - q)} \leq 1\}.$$

Получим верхние и нижние оценки суммы по Минковскому $\mathcal{E}_-,\, \mathcal{E}_+$ конечного числа эллипсоидов $\mathcal{E}_1, \ldots, \mathcal{E}_n$:
$$\mathcal{E}_- \subset \mathcal{E}_1 + \ldots + \mathcal{E}_n \subset \mathcal{E}_+.$$

\subsection{Внешняя оценка}
Пусть $p_1, \ldots, p_n > 0$. Покажем, что 
$$\mathcal{E}_+ = (p_1 + \ldots + p_n)\left(\dfrac{Q_1}{p_1} + \ldots + \dfrac{Q_n}{p_n}\right)$$
является внешней оценкой.

Действительно,
\begin{multline}\notag
\rho(l|\mathcal{E}_+)^2 = \Sum{i=1}{n}\Scal{l}{Q_il} + \sum_{i > j} \left(\dfrac{p_j}{p_i}\Scal{l}{Q_il} + \dfrac{p_i}{p_j}\Scal{l}{Q_jl}\right) \geq \\ \geq\Sum{i=1}{n}\Scal{l}{Q_il} + 2\sum_{i > j}\sqrt{\Scal{l}{Q_il}\Scal{l}{Q_jl}} = \rho(l|\mathcal{E}_1 + \ldots + \mathcal{E}_n)^2.
\end{multline}

Равенство здесь достигается тогда и только тогда, когда $p_i = \sqrt{\Scal{l}{Q_il}},\ i = 1, \ldots, n$. Таким образом, опорный вектор $\mathcal{E}_+$ по направлению $l$ совпадает с опорным ветором $\mathcal{E}_1 + \ldots + \mathcal{E}_n$, и потому
$$ \mathcal{E}_1 + \ldots + \mathcal{E}_n = \bigcap_{l \in S_1} \mathcal{E}_+(l).$$

\subsection{Внутренняя оценка}

Пусть матрица конфигурации для $\mathcal{E}_-$
$$Q_- = Q_*^*Q_*, \quad Q_* = \Sum{i=1}{n}S_iQ_i^\frac12,$$
где $S_i$ --- ортогональные матрицы.

С помощью неравенства Коши--Буняковского можно показать, что $Q_-$ действительно является внутренней оценкой, и, более того $\rho(l|\mathcal{E}_-) = \rho(l|\mathcal{E}_1 + \ldots + \mathcal{E}_n)$ в том и только том случае, когда $S_iQ_i^\frac12l = \lambda_iS_1Q_1^\frac12l$.

\section{Эллипсоидальные оценки для интеграла}
Рассмотрим выражение 
$$I(t) = \mathcal{E}_0(0, Q_0) + \int\limits_{t_0}^{t}\mathcal{E}(0, Q(\tau))d\tau$$
--- интеграл от эллипсоида $\mathcal{E}(0, Q(t)))$ по переменной $\tau \in [t_0, t]$. Введем разбиение $\{\tau_i\}_{i = 1}^{N}$ отрезка интегрирования $[t_0, t]$ и будем представлять интеграл в виде сумм 
$$
I(t) = \lim\limits_{N \rightarrow \infty}I_N,
$$
где $I_N = \mathcal{E}_0 + \sum\limits_{i = 1}^{N}\sigma \mathcal{E}_i,\quad \sigma = \dfrac{t - t_0}{N}, \quad \mathcal{E}_{i} = \mathcal{E}(0, Q(\tau_i)).
$

\subsection{Внешняя оценка}
Матрица конфигурации внешней оценки интегральной суммы находится по формуле
\begin{multline}\notag
Q_+^N = \left( p_0 + \sum\limits_{i = 1}^{N}p_i \right)\left( \dfrac{Q_0}{p_0} + \sum\limits_{i = 1}^{N} \dfrac{Q_i \sigma^2}{p_i}\right) = \left\{ p_0 = \langle l, Q_0 l \rangle ^{\frac{1}{2}},\ p_i = \sigma \langle l, Q_i l \rangle ^{\frac{1}{2}}\right\} =\\
= \left( p_0 + \sum\limits_{i = 1}^{N}\sigma\langle l, Q_il\rangle^{\frac{1}{2}} \right)\left( \dfrac{Q_0}{p_0} + \sum\limits_{i = 1}^{N} \sigma \dfrac{Q_i}{\langle l, Q_il\rangle^{\frac{1}{2}}}\right).
\end{multline}

При стремлении $N$ к бесконечности получим следующую матрицу конфигурации для внешней оценки интеграла:
$$Q(\tau) = \left( p_0 + \int\limits_{t_0}^{t} p(\tau)d\tau\right)\left( \dfrac{Q_0}{p_0} + \int\limits_{t_0}^{t}\dfrac{Q(\tau)}{p(\tau)}d\tau\right).$$

Таким образом, интеграл аппроксимируется пересечением эллипсоидов по различным направлениям $l$, то есть $I(t) = \bigcap\limits_{\Norm{l} = 1} \mathcal{E}_+(l)$.

\subsection{Внутренняя оценка}
Аналогичными рассуждениями приходим к тому, что матрица конфигурации нижней оценки для интеграла имеет вид
$$Q_- = Q_*^*Q_*, \quad Q_* = S_0Q_0^\frac12 + \Int{t_0}{t}S(\tau)Q^\frac12(\tau)d\tau,$$

причем оценка будет точной по направлению $l$, если 
$$S_0Q_0^\frac12l = \lambda(\tau)S(\tau)Q^\frac12(\tau)l.$$

В качестве $S_0$ далее будем использовать единичную матрицу.

\section{Эллипсоидальное оценивание множества разрешимости}
Будем дополнительно предполагать, что в задаче (\ref{task}) выполнено $\ B(t)Q(t)B^*(t) > 0$ при любых $t \in [t_0,\, t]$. Оценим множество разрешимости $\mathcal{W}[t]$.

Найдем траекторию $x(t)$ по формуле Коши:

$$x(t) = X(t, t_1)x(t_1) + \int\limits_{t_1}^{t} X(t, \tau)B(\tau)u(\tau)d\tau,$$

где $X(t, \tau)$ --- фундаментальная матрица, полученная как решение задачи Коши:

\begin{equation}\label{fund}
\left\{
\begin{aligned}
&\dfrac{\partial X(t, \tau)}{\partial \tau} = -X(t, \tau)A(\tau),\\
&X(t, t) = I.
\end{aligned}
\right.
\end{equation}



Множество разрешимости можно представить следующим образом:
$$
\mathcal{W}(t, t_1, \mathcal{E}(x_1, X_1)) = X(t, t_1)\mathcal{E}(x_1, X_1) - \int\limits_{t}^{t_1}X(t, \tau)B(\tau)\mathcal{E}(p(\tau), P(\tau))d\tau.
$$

\begin{St}
Пусть $\mathcal{E}(p, P)$ --- эллипсоид с центром $p$ и матрицей конфигурации $P$, a $B$ --- невырожденная квадратная матрица тогда $B\mathcal{E}(p, P) = \mathcal{E}(Bp, BpB^*)$.
\end{St}
\begin{Proof}
Пусть $x \in \mathcal{E}(p, P)$, что равносильно $\Scal{x}{P^{-1}x}  \leq 1$.

Рассмотрим теперь $B\mathcal{E}(p, P) = \mathcal{E}(q, Q)$, для него будет верно 
$$\langle B^{-1}x, P^{-1}B^{-1}x\rangle = \langle x, \left(B^{-1}\right)^* P^{-1} B^{-1}x \rangle \leqslant 1.$$
Следовательно, $Q^{-1} = \left(B^{-1}\right)^* P^{-1} B^{-1}$ и $Q = B P B^*$, что и требовалось доказать.
\end{Proof}


Тогда множество $\mathcal{W}[t]$ может быть записано в виде:
\begin{multline}\notag
\mathcal{W}[t] = \mathcal{E}(X(t, t_1)x_1,\, X(t, t_1)X_1 X^*(t, t_1)) - \\-\int\limits_{t}^{t_1} \mathcal{E}(X(t, \tau)B(\tau) p(\tau),\, X(t, \tau) B(\tau) P(\tau) B^*(\tau) X^*(t, \tau))d\tau = \\ = \mathcal{E}(X(t, t_1)x_1,\, X(t, t_1)X_1 X^*(t, t_1)) +\int\limits_{t_1}^{t} \mathcal{E}(X(t, \tau)p_B(\tau),\, X(t, \tau) P_B(\tau) X^*(t, \tau))d\tau.
\end{multline}

Здесь $p_B(\tau) = B(\tau) p(\tau), \, P_B(\tau) = B(\tau) P(\tau) B^*(\tau).$

Для построения внешних и внутренних оценок множества $\mathcal{W}[t]$ воспользуемся внешними и внутренними оценками для интеграла, которые мы получили выше.

\subsection{Внешняя оценка}
Пусть $\mathcal{E}(q_{+}(t), Q_{+}(t))$ - внешняя эллипсоидальная оценка множества $\mathcal{W}[t]$. Параметры задаются формулами
$$
q_+ = X(t, t_1)x_1 + \Int{t_1}{t}X(t, \tau)B(\tau)q(\tau)d\tau,
$$

$$
Q_{+}(t) = \left( p_0 - \int\limits_{t_1}^{t} p(\tau)d\tau\right)\left(\dfrac{X(t, t_1)X_1 X^*(t, t_1)}{p_0} - \int\limits_{t_1}^{t} \dfrac{X(t, \tau)B(\tau)P(\tau) B^*(\tau) X^*(t, \tau)}{p(\tau)}d\tau\right),
$$ 

где 
$$
\begin{aligned} 
&p_0 = \langle l, X(t, t_1)X_1 X^*(t, t_1)l \rangle^{\frac{1}{2}},\\
&p(\tau) =\langle l, X(t, \tau) B(\tau) Q(\tau) B^*(\tau) X^*(t, \tau) l\rangle^{\frac{1}{2}}.
\end{aligned}
$$

Выбранные таким образом $p$ позволяют получить оценку, точную в направлении $l$.
\subsection{Внутренняя оценка}
Теперь получим внутреннюю эллипсоидальную оценку $\mathcal{E}(q_{-}(t), Q_{-}(t))$.

Будем искать матрицу конфигурации в виде:
$${X}_-(t) = Q_*^*(t)Q_*(t),$$
$$\text{где } Q_*(t) = X_1^\frac12X^*(t, t_1) + \Int{t}{t_1}S(\tau)P^{\frac12}(\tau)B^*(\tau)X^*(t_1, \tau)d\tau.$$

Эта оценка точна по направлению $l$, если для любых $\tau$ будет выполнено:
\begin{equation}\label{eq_cond}
\lambda(\tau)X_1^\frac12X^*(t, t_1)l = S(\tau)З^{\frac12}(\tau)B^*(\tau)X^*(t, \tau)l.
\end{equation}
Матрица $S(\tau)$ ортогональна, поэтому: 
$$\lambda(\tau) = \dfrac{\Norm{P^{\frac12}(\tau)B^*(\tau)X^*(t, \tau)l}}{\Norm{X_Q^\frac12X^*(t, t_1)l}}.$$

Найдем теперь $S(\tau)$ из сингулярного разложения векторов
$$a = \lambda(\tau)X_1^\frac12X^*(t, t_1)l = U_1d_1v_1,$$
$$b = P^{\frac12}(\tau)B^*(\tau)X^*(t, \tau)l = U_2d_2v_2.$$
Здесь $U_1, U_2$ --- ортогональные матрицы, $d_1 = (d_{11}, 0, \ldots, 0)^T,\, d_2 = (d_{21}, 0, \ldots, 0)^T$,
$v_1, v_2 \in \{-1,\, 1\}$. Так как нормы векторов совпадают, и $d_1 \geq 0, \, d_2 \geq 0$, верно $d1 = d2$.

Положим $S(\tau) = \dfrac{v_1}{v_2}U_1U_2^*$. Тогда $S(\tau)$ окажется ортогональной, равенство (\ref{eq_cond}) будет выполнено.

\section{Описание алгоритма}
\begin{enumerate}
    \item Зададим входные данные.
    \item Зададим количество направлений N. Сгенерируем N направлений $l_0$, равномерно разбивающих окружность в плоскоти $(l_1, l_2)$.
    \item Найдем фундаментальную матрицу, как функцию $X(t, \cdot)$, из системы (\ref{fund}).
    \item По формулам из Раздела 4 вычислим внутренние и внешние аппроксимации для каждого направления $l_0$. Эти оценки являются точными по этому направлению.
    \item По каждой из оценок вычислим опорный вектор по направлению $l_0$. Множество разрешимости - объединение этих опорных векторов.
    \item Для построение аппроксимации трубки разрешимости в каждый момент времени $t$ будем отстраивать множество разрешимости.
\end{enumerate}

\section{Примеры работы алгоритма}

Сначала проиллюстрируем внешнюю и внутренню аппроксимацию суммы эллипсов $\mathcal{E}(q_1, Q_1), \\ \mathcal{E}(q_2, Q_2), \mathcal{E}(q_3, Q_3),$ где:
$$Q_1 = 
\begin{pmatrix}
  3& 2\\
  2& 2
\end{pmatrix}, \,
Q_2 = 
\begin{pmatrix}
  5& 1\\
  1& 2
\end{pmatrix},\,
Q_3 = 
\begin{pmatrix}
  1& -1\\
  -1& 6
\end{pmatrix},\,
q_1 = 
\begin{pmatrix}
  0\\
  0
\end{pmatrix},\,
q_2 = 
\begin{pmatrix}
  0\\
  0
\end{pmatrix},\,
q_3 = 
\begin{pmatrix}
  0\\
  2
\end{pmatrix}.
$$
\begin{figure}[h]
	\center
    \includegraphics[scale=0.6]{sum.png}
    \caption{Сумма эллипсов.}
\end{figure}

\begin{figure}
	\center
    \includegraphics[scale=0.4]{int.png}
    \caption{Внутренние оценки.}
\end{figure}

\begin{figure}
	\center
    \includegraphics[scale=0.4]{ext.png}
    \caption{Внешние оценки.}
\end{figure}

\newpage
Теперь рассмотрим систему:
$$
A(t) = 
\begin{pmatrix}
  cos(t)& -sin(t)& 0\\
  sin(t)& cos(t)& 0\\
  0& 1& 0
\end{pmatrix}, \,
B(t) = 
\begin{pmatrix}
  cos(t)& -sin(t)& 0\\
  sin(t)& cos(t)& 0\\
  0& 1& 0
\end{pmatrix}, \,
P(t) = 
\begin{pmatrix}
  |cos(t)|+1& 0& 0\\
  0& |cos(t)|+1& 0\\
  0& 0& 1
\end{pmatrix},
$$
$$
p(t) = 
\begin{pmatrix}
  t\\
  0\\
  0
\end{pmatrix}, \,
X_1 = 
\begin{pmatrix}
  6& 1& 5\\
  1& 1& 2\\
  5& 2& 13
\end{pmatrix}, \,
x_1 = 
\begin{pmatrix}
  2\\
  1\\
  1
\end{pmatrix}, \,
t_1 = 10.
$$
\begin{figure}[h]
	\center
    \includegraphics[scale=0.4]{two.png}
    \caption{Проекция множества разрешимости на на координатную плоскость $Ox_2x_3$.}
\end{figure}

\begin{figure}
	\center
    \includegraphics[scale=0.3]{three.png}
    \caption{Множество разрешимости в момент $t = 5$.}
\end{figure}
\begin{figure}
	\center
    \includegraphics[scale=0.3]{forth.png}
    \caption{Множество разрешимости в момент $t = 10$.}
\end{figure}
\end{document}